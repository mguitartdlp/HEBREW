\documentclass[12pt]{article}

\usepackage{polyglossia}
\setmainlanguage{catalan}
\setotherlanguage{hebrew}
\newfontfamily\hebrewfont[Script=Hebrew]{FreeSerif}


\begin{document}

Aquest és un text en català i hebreu.

Per LuaLatex.

\begin{otherlanguage}{hebrew}
זהו טקסט בעברית.
\end{otherlanguage}

Aquest és un altre text en català.

#exemple de taula català/hebreu centrada i amb títol al peu
\begin{table}[h]
\centering
\begin{tabular}{|c|c|}
\hline
\textbf{Hebreu} & \textbf{Català} \\
\hline
\selectlanguage{hebrew}הֶבֶל\selectlanguage{catalan} & Hebel \\
\hline
\selectlanguage{hebrew}שָׁבוּעַ\selectlanguage{catalan} & Sha-vu6t \\
\hline
\selectlanguage{hebrew}בָּחוּר\selectlanguage{catalan} & Bahur \\
\hline
\selectlanguage{hebrew}עוֹלָם\selectlanguage{catalan} & 'Olam \\
\hline
\selectlanguage{hebrew}יִתְרוֹן\selectlanguage{catalan} & Yitrón \\
\hline
\selectlanguage{hebrew}עָמָל\selectlanguage{catalan} & 'Amal \\
\hline
\end{tabular}
\caption{Paraules en hebreu i el seu equivalent en català}
\end{table}

#Per escriure en hebreu en el mig del document (d'esquerra a dreta)
\selectlanguage{hebrew}
הֶבֶל % Hebel

שָׁבוּעַ % Sha-vu6t

בָּחוּר % Bahur

עוֹלָם % 'Olam

יִתְרוֹן % Yitrón

עָמָל % 'Amal

\selectlanguage{english}

\end{document}
